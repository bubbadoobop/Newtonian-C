\documentclass[]{article}
\usepackage{lmodern}
\usepackage{amssymb,amsmath}
\usepackage{ifxetex,ifluatex}
\usepackage{fixltx2e} % provides \textsubscript
\ifnum 0\ifxetex 1\fi\ifluatex 1\fi=0 % if pdftex
  \usepackage[T1]{fontenc}
  \usepackage[utf8]{inputenc}
\else % if luatex or xelatex
  \ifxetex
    \usepackage{mathspec}
  \else
    \usepackage{fontspec}
  \fi
  \defaultfontfeatures{Ligatures=TeX,Scale=MatchLowercase}
\fi
% use upquote if available, for straight quotes in verbatim environments
\IfFileExists{upquote.sty}{\usepackage{upquote}}{}
% use microtype if available
\IfFileExists{microtype.sty}{%
\usepackage{microtype}
\UseMicrotypeSet[protrusion]{basicmath} % disable protrusion for tt fonts
}{}
\usepackage[unicode=true]{hyperref}
\hypersetup{
            pdfborder={0 0 0},
            breaklinks=true}
\urlstyle{same}  % don't use monospace font for urls
\IfFileExists{parskip.sty}{%
\usepackage{parskip}
}{% else
\setlength{\parindent}{0pt}
\setlength{\parskip}{6pt plus 2pt minus 1pt}
}
\setlength{\emergencystretch}{3em}  % prevent overfull lines
\providecommand{\tightlist}{%
  \setlength{\itemsep}{0pt}\setlength{\parskip}{0pt}}
\setcounter{secnumdepth}{0}
% Redefines (sub)paragraphs to behave more like sections
\ifx\paragraph\undefined\else
\let\oldparagraph\paragraph
\renewcommand{\paragraph}[1]{\oldparagraph{#1}\mbox{}}
\fi
\ifx\subparagraph\undefined\else
\let\oldsubparagraph\subparagraph
\renewcommand{\subparagraph}[1]{\oldsubparagraph{#1}\mbox{}}
\fi

\date{}

\begin{document}

\section{Newtonian-C, by Ian
Mitchell}\label{newtonian-c-by-ian-mitchell}

When using Newtonian-C, the first thing you will see is a menu, which,
is pretty self explanatory.

\begin{verbatim}
Newtonian-C, by Ian Mitchell (c) 2017, Gnu GPL v3
Please pick an option from below 
-------------------------------------------------
 1) Kinematic system
 2) Wave system
 3) Gravitational System
 4) About

 --->
\end{verbatim}

All you need to do is pick an option on the menu, and then add the
neccesary values. For the sale of simplicity on this documentation,
let's choose option 2 (Wave system), since it only has one value needed
to operate.

\begin{verbatim}
Newtonian-C, by Ian Mitchell (c) 2017, Gnu GPL v3
Please pick an option from below 
-------------------------------------------------
 1) Kinematic system
 2) Wave system
 3) Gravitational System
 4) About

 ---> 2
\end{verbatim}

Upon choosing option number 2, and pressing ENTER (RETURN), you should
see a screen that looks something like this\ldots{}

\begin{verbatim}
Newtonian-C, by Ian Mitchell (c) 2017, Gnu GPL v3
Please pick an option from below 
-------------------------------------------------
 1) Kinematic system
 2) Wave system
 3) Gravitational System
 4) About

 ---> 2
 Please enter the Period (time) of the wave  
 >
\end{verbatim}

Next, all you need to do is input the value needed. Let's say, 10.

\begin{verbatim}
Newtonian-C, by Ian Mitchell (c) 2017, Gnu GPL v3
Please pick an option from below 
-------------------------------------------------
 1) Kinematic system
 2) Wave system
 3) Gravitational System
 4) About

 ---> 2
 Please enter the Period (time) of the wave  
 > 10
  Period: T = 10.000000                     
  Frequency: f = 0.100000                       
  Angular Frequency: ω = 0.628318                   
  Amplitude = 100.000000                        
  Velocity: v = 6.283184                        
  Wavelength: λ = 62.831841
\end{verbatim}

It will then output all the values calculated by the program.
\begin{center}
IMPORTANT
\end{center}

\begin{center}\rule{0.5\linewidth}{\linethickness}\end{center}

When using options 1 or 3, it will tell you to enter the values in the
exact order it says. Falure to do so will result in misleading outputs,
that are wrong in the context of whatever calculations that you may be
making.

\end{document}